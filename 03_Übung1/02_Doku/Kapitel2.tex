\chapter{Ausgeben von Signalen}
In dieser Aufgabe sollte der DSP ein Sinussignal generieren. Die Frequenz sollte 
dabei per Taster in 200Hz Schritten einstellbar sein. 
Die drei auf dem DSP angebrachten LEDs sollten die Frequenzen 200Hz, 400Hz und 4kHz darstellen.

Zur Realisierung dieser Aufgabe wurde in der Vorbereitung ein Funktion genSinus erstellt, 
die bei jedem Aufruf den nächsten Wert des Sinussignals ausgibt.

\begin{adjustbox}{width=\textwidth,keepaspectratio}
\lstinputlisting[title=gensinus.c]{gensinus.c}
\end{adjustbox}

Die Funktion ist so Implementiert, dass sie den jeweiligen Zeitpunkten einen entsprechenden Sinuswert zuweist und diesen zurück weißt. 
Als \"Ubergabeparameter sind zum einen die Amplitude und zum anderen die Frequenz vorgesehen wobei die Frequenz in 200Hz Schritten \"ubergeben wird. 

Das neue \begin{math}\omega_{Norm}\end{math}, also wird entsprechend mit 
\begin{equation}\label{normierteKreisfrequenz}
 \omega_{Norm}=2\pi*f 
\end{equation}  
ermittelt und normiert.\\\par
Der Aufruf durch den Timer-Interrupt stellt dabei die Periodizität sicher. 
Das alte \begin{math}\omega_{Norm}\end{math} wird dann mit dem neuen \begin{math}\omega_{Norm}\end{math} addiert und mit der Funktion sin aus math.h 
wird der entsprechende Sinuswert ermittelt. Die Skalierung erfolgt durch multiplikation mit A, da sin() einen Normierten Wert zwischen 0 und 1 zurück gibt.

Um die LEDs und Taster entsprechend nutzen zu können wurde die main.c angepasst.
\begin{adjustbox}{width=\textwidth,keepaspectratio}
\lstinputlisting[title=main.c]{main2.c}
\end{adjustbox}

Außerdem wurde über die programmable Flags der Interrupt der Taster erlaubt, sodass ein Tastendruck die isr aufruft. 
Des weiteren musste entsprechend die isr angepasst werden.

\begin{adjustbox}{width=\textwidth,keepaspectratio}
\lstinputlisting[title=isr.c]{isr2.c}
\end{adjustbox}