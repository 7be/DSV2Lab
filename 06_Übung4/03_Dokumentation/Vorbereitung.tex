\chapter{Vorbereitung}
\todo{Nic überprüfe das bitte}
In folgender Tabelle sind die DTMF-Frequenzen sowie die normierten Kreisfrequenzen (Abtastfrequenz: 48kHz) zu sehen.\\

\begin{table}[H]
	
\centering
\begin{tabular}{l|l|l|l|l}
	Hz & 1209 & 1336 & 1477 & 1633 \\
	\begin{math}\Omega_n\end{math}  & 0,158257 & 0,174881 & 0,193338 & 0,213759\\
	\hline\hline 
	697 & & & & \\
	0,091237 & 1 & 2 & 3 & A \\
	\hline
	770 & & & & \\
	0,091237 & 4 & 5 & 6 & B \\
	\hline
	852 & & & & \\
	0,091237 & 7 & 8 & 9 & C \\
	\hline
	941 & & & & \\
	0,091237 & * & 0 & \# & D \\
	\hline
\end{tabular}\\
\end{table}

Die normierten Kreisfrequenzen errechnen sich nach folgender Formel:


\begin{equation}
  \Omega_n=\frac{2*\pi*f_{DTMF}}{48kHz}
\end{equation}

Nun soll mit folgenden Formeln die Ordnung der FFT bestimmt werden. 
\begin{equation}
  \Delta\Omega=\frac{2*\pi}{N}=2*\Pi*\frac{\Delta f}{f_T}
\end{equation}
Daraus folgt:
\begin{equation}
  N=\frac{f_T}{\Delta f}
\end{equation}
Die minimalste Frequenz \begin{math}\Delta f\end{math} befindet sich zwischen den DTMF-Frequenzen 697Hz und 770Hz.
Auch sollen laut Vorgabe 3 weitere St\"utzstellen verwendet werden, das macht insgesamt 4.
Daraus ergibt sich folgende Filterordnung.
\begin{equation}
  N=\frac{48kHz}{0,25*(770Hz-697Hz)}=2630,14
\end{equation}
Da wir eine Funktion aus der DSP Library verwenden bei der eine 2er Potenz von N ben\"otigt wird, m\"ussen wir ebenfalls eine Filterordnung mit 2er Potenz verwenden.
Dies macht eine Filterordnung von N=4096.