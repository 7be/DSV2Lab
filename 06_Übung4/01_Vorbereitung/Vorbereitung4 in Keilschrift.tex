In folgender Tabelle sind die DTMF-Frequenzen sowie die normierten Kreisfrequenzen (Abtastfrequenz: 48kHz) zu sehen.
 \begin{tabular}{l|p{3cm}r}
  Hz \newline \Omega_n  & 1209Hz \newline 0,158257 & 1336Hz \newline 0,174881 & 1477Hz \newline 0,193338 & 1633Hz \newline 0,213759 \\
  697Hz \newline 0,091237 & 1 & 2 & 3 & A \\
  770Hz \newline 0,100792 & 4 & 5 & 6 & B \\
  852Hz \newline 0,111526 & 7 & 8 & 9 & C \\
  941Hz \newline 0,123176 & * & 0 & # & D \\
 \end{tabular}
Die normierten Kreisfrequenzen errechnen sich nach folgender Formel:
\begin{equation}
  \Omega_n=\frac{2*\pi*f_DTMF}{48kHz}
  \end{equation}

Nun soll mit folgenden Formeln die Ordnung der FFT bestimmt werden. 
\begin{equation}
  \Delta\Omega=\frac{2*\pi}{N}=2*\Pi*\frac{\Delta f}{f_T}
\end{equation}
Daraus folgt:
\begin{equation}
  N=\frac{f_T}{\Delta f}
\end{equation}
Die minimalste Frequenz \Delta f befindet sich zwischen den DTMF-Frequenzen 697Hz und 770Hz.
Auch sollen laut Vorgabe 3 weitere St\"utzstellen verwendet werden, das macht insgesamt 4.
Daraus ergibt sich folgende Filterordnung.
\begin{equation}
  N=\frac{48kHz}{0,25*(770Hz-697Hz)}=2630,14
\end{equation}
Da wir eine Funktion aus der DSP Library verwenden bei der eine 2er Potenz von N ben\"otigt wird, m\"ussen wir ebenfalls eine Filterordnung mit 2er Potenz verwenden.
Dies macht eine Filterordnung von N=4096.